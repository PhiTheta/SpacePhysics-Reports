%%%%%%%%%%%%%%%%%%%%%%%%%%%%%%%%%%%%%%%%%%%%%%%%%%%%%%%%%%%%%%%%%%%%%%%%
\section{Assignment}
\subsection{Overview}

\subsection{Results}

Data was downloaded from \url{http://sidc.be/silso/datafiles} and \url{ftp://ftp.ngdc.noaa.gov/STP/GEOMAGNETIC_DATA/INDICES/KP_AP}, year 2003, and the data format was checked.
\subsection{Discussion}




%%%%%%%%%%%%%%%%%%%%%%%%%%%%%%%%%%%%%%%%%%%%%%%%%%%%%%%%%%%%%%%%%%%%%%%%
\section{Assignment}
\subsection{Overview}
In this Assignment we modify the settings for the model IRI2001 with the data gathered in the previous task, and re-run the IRI2001 model.
The results are then compared to the default case.
\subsection{Results}

\begin{figure}[h]
	\centering
	\includegraphics[width=\linewidth]{images/E_number_density_plot1}	
	\caption{Electron number density Plot Old}
	\label{fig:ass2Plot1}
\end{figure}
\begin{figure}[h]
	\centering
	\includegraphics[width=\linewidth]{images/E_number_density_plot2}	
	\caption{Electron number density Plot New}
	\label{fig:ass2Plot2}
\end{figure}

\subsection{Discussion}
For the default case, the number density is much lower at higher altitudes than it is for the corrected value, for example: at an altitude of approx. 500km the number density is around $10^{5.05}$, whereas for the latter case the number density at 500km is approx. $10^{5.2}$.





%%%%%%%%%%%%%%%%%%%%%%%%%%%%%%%%%%%%%%%%%%%%%%%%%%%%%%%%%%%%%%%%%%%%%%%%
\section{Assignment}
\subsection{Overview}
The model IRI2001 used in Assignment 1 is used again. The year shall stay the same, but other parameters as Day of Year, the daily $F_{10.7}$ and the 12 month average sunspot number are changed.
\subsection{Results}
We used the following values for the parameters to be changed
\todo{make a table here or something}

year 2003
SD:7.1
average sunspot number 99.3
Selecting Day 1

\begin{figure}[h]
	\centering
	\includegraphics[width=\linewidth]{images/ass3plot1}	
	\caption{Electron number density with standard deviation}
	\label{fig:ass3Plot1}
\end{figure}

The sunspot number parameter has almost no effect at lower altitudes, however at approx. 250km the electron number density starts to vary. The lower the sunspot number the lower the electron number density at each respective altitude above 250km.
\todo{insert plot here km/e-}

2. The electron number density varies on a diurnal basis, but only very minutely. 

3. Comparing Winter and Summer (day 1 and day 181), the electron number density varies on a larger scale.
\subsection{Discussion}



%%%%%%%%%%%%%%%%%%%%%%%%%%%%%%%%%%%%%%%%%%%%%%%%%%%%%%%%%%%%%%%%%%%%%%%%
\section{Assignment}
\subsection{Overview}
\subsection{Results}
Since the given latitude and longitude are located in the auroral region, and 2003 was close to a maximum of the sun's 11-year cycle, the IRI2001 Model is not suitable as this is not magnetically-quite region.
\subsection{Discussion}



%%%%%%%%%%%%%%%%%%%%%%%%%%%%%%%%%%%%%%%%%%%%%%%%%%%%%%%%%%%%%%%%%%%%%%%%
\section{Assignment}


For this assignment, the year 2012 was utilized  in order to determine the ionospheric layers. The corresponding day of the year, Daily $F_{10.7}$, and Sunspot Number were 270, 139.9, and 84.5, respectively. The coordinating grid was kept the same as previous assignments. The figure below displays the number density of various particles and ions as the altitude is varied. 

	\begin{figure}[h!]
		\centering
		\includegraphics[width=\linewidth]{images/ass5_properties_plot}
		\caption{Altitude vs. Particle Number Density ($cm^{-3}$)}
	\end{figure}

		\subsection{Discussion}
		Based on the densities shown in the particle density graph, it can be seen that D-layer exists between ~50-100 km. The presence of low amounts of NO ions corroborates the aforementioned statement as only a high-wavelength spectral line would be able to penetrate to this level and ionize the heavy ions. 
		\par
		The E-layer is somewhere between ~95-150 km. This layer consists of high amounts of $0_2^+$ and $N_2^+$ ions which are produced when x-rays and ultraviolet rays dissociate $O_2$ and $N_2$ molecules.
		\par
		The F layers is divided into the F1 and F2 layers. The F1 layer exists from between ~150-220 km. This can be determined by seeing the number density of electrons and $O^+$. Electron density is somewhere between $5e5$ to $5e6$ and $O^+$ density is higher due to the lighter particle floating to the higher regions. The F2 layer is between ~200-500 km. The lighter ions such as $He_+, N^+, and H^+$  exist in this layer due to their light weight.
		\par
		The Chapman layer differs from IRI model in displaying the number densities of particles especially in the F2 layer. These differences exist because the Chapman layer takes into account some parameters that the IRI model does not and vice versa. For example, the Chapman layer takes into account the Sun's zenith angle. In addition, the chapman layer method assumes that the radiation from sun is monochromatic, the atmosphere consists of only one gas, etc. These assumptions allow the chapman layer to create a "better density profile especially for the topside ionosphere (Jin)".


% References

% http://center.shao.ac.cn/geodesy/publications/Jin_2007EPS.pdf
% http://nova.stanford.edu/~vlf/IHY_Test/Tutorials/TheIonosphere/IonosphericMorphology.pdf



%\subsection{Overview}
%\subsection{Results}
%\todo{todo}
%D-Layer below 90
%
%E 90-130
%
%F1-130
%\begin{figure}[h]
%	\centering
%	\includegraphics[width=\linewidth]{images/ass5_properties_plot}	
%	\caption{bla}
%	\label{fig:ass5Plot}
%\end{figure}
%
%
%\subsection{Discussion}

